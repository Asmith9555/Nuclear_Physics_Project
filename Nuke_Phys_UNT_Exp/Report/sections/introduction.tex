\documentclass[../final_report_main.tex]{subfiles}
\begin{document}
A common nuclear process seen in radioactive sources involves the emission of
two gamma rays in coincidence (simultaneously). A source that undergoes this
phenomenon frequently is \textsuperscript{22}Na. \textsuperscript{22}Na exhibits
this process through $\beta$+ decay producing a positronium (positron electron
pair) which annihilates, emitting two gamma rays. These gamma rays in theory
have equal energies of 511 keV and a net momentum that is conserved. The
momentum of the positronium on paper is zero, therefore the direction of each
photon should be 180 degrees from the other to conserve momentum. The goal of
this experiment is too prove the angular correlation between these two
coincident gammas.
\end{document}
