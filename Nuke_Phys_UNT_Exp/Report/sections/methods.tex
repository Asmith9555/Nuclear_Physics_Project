\documentclass[.../final_report_main.tex]{subfiles}
\begin{document}
The angle of correlation between the coincident gammas, in theory is 180
degrees. This experiment tested this theory through comparing two different
models: the amount of coincidence counted at a given off-set (measured from
180 degrees) angle and a theoretical geometric model of coincident gammas with
180 degree correlation. The general design of the experiment involved two
gamma-ray detectors in a time-gate circuitry wired to only measure gammas that
hit both detectors within a short period of time between the other (Figure
\ref{figure:circuitry}). These detectors were set equidistant (varied from 15 cm
to 30 cm) from a source of NA-22 where one detector was held still, while the
other was rotated freely around the source (Figure \ref{figure:set_up}).
For the first model the angle of the detector was varied in increments of 1-2
degrees offset from 180 and a spectrum of coincident gamma rays was collected
for 120-240 second at each angle. This spectrum was integrated for each angle
to get the desired net count rate (Figure \ref{figure:spectrum}).
The second model aims to compares the overlapping area between detectors with
the net count rate per angle. If coincident gammas eject directly opposite from
each other, in order for any coincidence to be measured, part of the face of
each detector must overlap the other. This is essentially a measure of the
set-ups ability to count coincidence and the general trend can be compared with
the first set of data to see if the theory is supported.
\end{document}
