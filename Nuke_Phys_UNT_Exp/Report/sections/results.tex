\documentclass[../final_report_main.tex]{subfiles}
\begin{document}
Table \ref{table:model1_table} diplays the measured count rates at their
respective offset angles from 180 degrees. The trend turned out to be similar
for both distances from the source showing a maximum count rate somewhere
between -1 to 1 degrees, and following a Gaussian curve as the offset angle
increased from the 180 degree angle. (Figure \ref{figure:model1_graph}).
For the second model, the overlapping area of the detectors as a function of
off-set angle, the geometry in Figure \ref{figure:geometry} was used. To start
this derivation, the the total area overlapped was set as a linear combination
of the area of the two inner circles $A_{1}$ \& $A_{2}$ (Figure
\ref{figure:geometry}). The areas of these pieces follow as:
%Equation Blocks
\begin{equation}
  \begin{gathered}
   A_{1} = 2\int_{d1}^{r1}\sqrt{r_{1}^{2}-x^{2}}dx \nonumber\\
   A_{2} = 2\int_{d-r_{2}}^{d1}\sqrt{r_{2}^{2}-(x-d)^{2}}dx \nonumber
  \end{gathered}
\end{equation}
After substitution and simplification we are left with:
\begin{equation}
  \begin{aligned}
  A_{total}=r_{1}^{2}cos^{-1}(\frac{d_{1}}{r_{1}})−d_{1}\sqrt{r_{1}^{2}−d_{1}^
  {2}}\\+r_{2}^{2}cos^{-1}(\frac{d_{2}}{r_{2}})−d_{2}\sqrt{r_{2}^{2}−d_{2}^{2}}
  \nonumber
  \end{aligned}
\end{equation}
where...
\begin{equation}
  \begin{aligned}
  d_{1} = \frac{r_{1}^{2}-r_{2}^{2}+d^{2}}{2d} \nonumber \\ d_{2} =
  \frac{r_{2}^{2}-r_{1}^{2}+d^{2}}{2d} \nonumber \\ d = \imath \cdot sin(\theta)
  \end{aligned}
\end{equation}
This formula only holds true for when $r_{1}\geq r_{2}$ and $d \geq r_{2} -
r_{1}$. Assertion 1 is always true in our set-up, however due to Assertion 2,
the correlation falls apart when the offset angle from 180 degrees is small.
These points were simply calculated at the area of the detector of smaller
radius. Once all of the overlapping areas were calculated (Table
\ref{table:model2_table}), we plotted and also fit the data with a Gaussian
function. While the fit had a bad confidence interval, the correlation between
the two models doesn't hinge on direct correlation. The fact that both sets of
data follow the general trend of having a maximum at 0 degrees, and a steep
decrease in area as the angle sweeps in either direction from this maximum, is
enough to point towards a 180 degree correlation. (Figure
\ref{figure:model1_graph} and \ref{figure:model2_graph}).
\end{document}
